\documentclass[11pt,a4paper,oneside,onecolumn]{article}
\usepackage[ruled,vlined]{algorithm2e}
\usepackage{minted}
\usepackage{booktabs}
\usepackage{multicol}
\usepackage{cleveref}


\begin{document}

\section{Information about inventory}

Information about an item-code in the inventory is categorised into the following two classes
\begin{multicols}{2}
	
\begin{minted}{python}

class BasicInfo:
    __name:  str 
    __price: float
     
\end{minted}

\vfill

\begin{minted}{python}
class Offer:
    __running: bool
    __n_items: np.uint64
    __coeffs: np.ndarray
    __reduction: float
    
\end{minted}

\end{multicols}

Among the Offer class members,
\begin{itemize}
\item{
\_\_running is True if there is an offer for the given item-code, or False otherwise.
}
\item{\_\_n\_items is the frequency at which the offer becomes eligible}
\item{\_\_coeffs is a two element array that contains information to effectively 
	apply the offer reduction.

In the given objective, offer or no-offer on item-codes can be casted as 

\begin{equation}
\_\_\mathrm{reduction}(p) = \_\_\mathrm{coeffs[0]}\times p+\_\_\mathrm{coeffs[1]})
\end{equation}

where $p$ is the price of an item and \_\_reduction is the value deducted from the
 item-code items total price whenever the offer becomes eligible}

\begin{table}[h]
\begin{center}
\begin{tabular}{ c c c c c }
 \toprule
 Code & Name & \_\_n\_items & \_\_coeffs[0] & \_\_coeffs[1] \\ 
 \midrule
 \midrule
 A & Apple & 3 & $1$ & $0$ \\  
 B & Banana & 3 & $3$ & $-100$ \\
 P & Pear & 0 & $0$ & $0$ \\
 \bottomrule
\end{tabular}
\end{center}
\caption{Inventory. Note that if \_\_n\_items for an item-code is zero in the inventory,
	it is interpreted as no offer.}
\label{tbl:inventory}
\end{table}

\end{itemize}

For convenience, FullInfo class is made-up of the above two classes, BasicInfo and Offer

\begin{minted}{python}
class FullInfo:
    basic_info: BasicInfo
    offer: Offer
\end{minted}

 An empty dictionary 
  \mint{python}|INVENTORY_ITEMS (: Dict[str, FullInfo]) = {}|
 is instantiated to hold the inventory information, and each row,
 information about an item-code, in \cref{tbl:inventory} is read and added to the dictionary.
 
 \begin{multicols}{2}
 \begin{minted}{python}
if (__n_items == 0):
     __running = False
else:
     __running = True
     \end{minted}
     
     
\vfill

\begin{minted}{python}
INVENTORY_ITEMS[Code] = FullInfo(
    BasicInfo(Name),
    Offer(__running, __n_items,
          __coeffs[0], __coeffs[1]
    )
 	
 \end{minted}
 \end{multicols}

Note that \_\_price(: BasicInfo) and \_\_reduction(: Offer) are set to $0.0$ by default 
if values are not specified.

When price is known, at a later point, for an item-code, the corresponding item-code price
and offer reductions are set using the respective class members as shown below

\begin{minted}{python}
class BasicInfo:
    def set_price(self, price: float):
        self.__price = price
\end{minted}

\begin{minted}{python}
class Offer:
    def set_reduction(self, price: float):
        self.__reduction = self.__coeffs[0] * price + self.__coeffs[1]	
\end{minted}

\section{Information about user collected items}
 
So far the data-structures discussed hold the inventory information.
The below Item class holds \_\_count (number of items user picked), 
and \_\_total (total price after applying the offer reductions) 
for an item-code.

\begin{minted}{python}
class Item:
    __count: uint64
    __total: float
\end{minted}

 An empty dictionary 
  \mint{python}|items (: Dict[str, Item]) = {}|
 is instantiated to hold the information about user picked items based on item-code.

Every time user scans an item-code, items[item-code] invokes \cref{algo:scan-item} to
perform necessary updates.
This way, \textbf{items} dictionary holds up-to-date count and total for all items
in the basket based on item-code.

\begin{algorithm}[H]
\DontPrintSemicolon
\SetAlgoLined
\caption{Check if eligible for offer at present}
\label{algo:offer-eligibility}
 \KwData{\;
  user\_n\_items(: uint64): Number of items that user has picked so far\;
  offer\_n\_items(: uint64): Frequency at which offer becomes eligible\;
  }
  
 \KwResult{offer\_eligible(: bool)}
  
 \eIf{count is zero}{
 set offer\_eligible to False \;
 }
 {
 set offer\_eligible to True, if count is an exact multiple of offer\_n\_items 
	(user\_n\_items \% offer\_n\_items == 0)\;
 }
\end{algorithm}

\begin{algorithm}[H]
\DontPrintSemicolon
\SetAlgoLined
\caption{Scan an item and update necessary details}
\label{algo:scan-item}

 \KwData{items: Item, item\_full\_info: FullInfo}
 \KwResult{Updated items (count and total)}
  
 Increase the scanned item count by one\;
 Increase the scanned item total by it's price\;
 Set offer\_eligible as the \cref{algo:offer-eligibility} return value\;
 \If{offer\_eligible}{
 Reduce the item total by offer reduction\;
 }
 
\end{algorithm}


The total price is obtained by looping over all item-codes in the items dictionary and 
accumulating each item-code total to the total.
\begin{equation}
\mathrm{total} = \sum_{\forall i}\mathrm{items}[i].\mathrm{get\_total()}
\end{equation}
	
\end{document}
